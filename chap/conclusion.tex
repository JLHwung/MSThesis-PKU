\specialchap{结论}
三角机制作为一个输出合成数据库的隐私机制无疑有着广泛的应用前景. 该合成数据库一方面保证了原始数据库的隐私, 另一方面对一大类统计查询 ------ 光滑查询提供了精确度保证. 更重要的是, 三角机制的时间代价关于原始数据的观测数的多项式, 而对目前已有的适用该问题的隐私机制, 其时间代价是超指数的. 对于高度光滑的查询, 三角机制保证了$O(n^{-1})$的精确度, 远远优于抽样误差$O\left(n^{-1/2}\right)$. 三角机制的实际运行时间还可以通过查询基与格点抽样来得到显著改善, 并通过隐私主成分分析来弥补部分损失的精确度. 在选定数据库上的实验结果支持三角机制的理论结果.

三角机制针对光滑查询可以给出精确的回答, 但现实中许多查询是非光滑的, 甚至是不连续的: 比如最常见的计数查询, 再如\parencite{blum2013learning}研究的矩形查询. 我们希望设计一种通用的隐私机制: 该机制可以回答除了光滑查询外, 还可以回答部分重要的离散查询的通用隐私机制, 该机制需要输出合成数据库, 还要尽可能高效. 如前所述, 对于简单的两参数查询, 输出合成数据库的隐私机制是指数时间的\parencite{ullman2011pcps}, 因此, 这个问题是非平凡的. 

最后, 三角机制要求数据库的域为$[-1, 1]^d$, 因而三角机制只能处理连续型的变量, 而对于离散型的多值分类变量, 是否可以用三角机制进行类似处理并保证一定的精度? 三角机制输出的合成数据库是连续的, 合成数据库应该如何进行离散化处理来保证离散型的变量体现在合成数据库中仍旧保持离散, 同时保证精度? 这些问题都是值得研究的方向. 