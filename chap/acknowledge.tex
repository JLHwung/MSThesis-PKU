\cleardoublepage\thispagestyle{empty}
\newgeometry{height = 240mm, width = 150mm, ignoreheadfoot, vcentering}
{
\vspace*{\fill}
\kaishu\zihao{4}
\centerline{致}
\centerline{我的父亲黄波、母亲王碧先}
\centerline{妹妹黄诗意以及 Dini}
\vskip 48ex
\vspace*{\fill}\par
}
\clearemptydoublepage\restoregeometry
\pdfbookmark[1]{致谢}{acknowledge}
\chapter*{致谢}
\markboth{致谢}{}
我在2013年春由姚远老师与贾金柱老师主持的统计学习课程中有机会听到王立威老师关于boosting与active learning的讲座, 王老师清晰的思维脉络与热情的演讲风格打动了我, 在随后的暑假中我顺利地进入了王老师组进行差分隐私的研究工作. 我诚挚地感谢王立威老师带领我走进差分隐私这个新兴的研究领域. 我时常有一些新鲜的想法, 但大部分在几个月后看起来都十分幼稚, 因而我要感谢王老师鼓励我继续探讨这些问题. 最后, 我通过自己的观察也体会到王老师对不同学生因材施教的用心良苦, 从工作分配、进度跟踪到及时反馈, 这一切都会十分花费精力, 因此, 我要感谢王老师对我们的悉心栽培. 

我感谢吴岚老师在本硕课程、指导本科生科研中的言传身教让我学会的一种思维方式: 从数学严格化的定义、定理中提炼出一种直观的描述. 通过这种思维方式, 我在研究生二年级逐渐明白了看论文与看课本的区别, 通过对论文的研习逐渐完成了从学习到研究的思维习惯转换. 尽管我日后未必会走上职业的学术道路, 但这种思维习惯是受益终身的. 

我感谢徐恺老师在指导我的本科毕业论文时对细节的严格要求. 从具体知识的严谨性到行文表达的规范性, 徐老师的教诲让我铭记于心. 我在本文的写作中也总是时刻提醒自己要注意每一个技术细节, 注意每一个标点符号. 严谨性的背后是一种高度的自律, 我在与徐老师在课上课下的接触中也无时不体会到这种自律带给人张弛有度的空间, 而这叫人愉悦. 因此, 我十分感谢徐老师对我独特的教导.

我感谢所有曾经教授过我的师长, 我不仅从他们的授课中更加容易地学习到自己想要的知识, 也从他们的言行所体现的个人魅力领悟到修养的重要. 

我感谢王子腾同学在我进入实验室这一年对我的热心帮助, 他是除去王立威老师以外第二个把我带进差分隐私这一研究领域并帮助我迅速跟上研究前沿的人. 本文的隐私主成分精确性定理是我与他的共同工作, 我感谢他在日常实验、讨论中分享给我的所有新鲜的想法, 这些想法通过互相的交流让我们的算法可以得到更好的改进. 

我感谢陈瑜希同学在算法上给我的许多颇具启发性的指导. 在我们用C++实现三角机制的时候, 她提出并实现了一种高速的组合遍历方法, 节约了大量实验时间. 除此之外, 她在编写程序时对键盘的效率使用也让我通过模仿而提高了自己的程序录入速度.

我感谢张佳琦同学与郑凯同学在讨论中提出许多独到的见解, 这使我们的讨论更加有收获. 我还要特别感谢张佳琦教会我正确的自由泳与哑铃姿势, 他们也促使我养成了每日运动的习惯, 这对于保持研究与工作所需要的旺盛精力是必要的. 

我感谢北京大学的每一个相交的同学, 我从他们各自不同志趣领悟到前路的宽广与未知. 每每念及同窗, 总让我面对前面未知的困难充满了征服的欲望.

我感谢我的家人, 包括 Dini, 对我无微不至的照顾. 我身在远方, 多有不周, 但他们仍然对我给予了支持与理解, 我感谢这样一个温暖的家庭带给我的柔软的爱意. 

最后, 我个人感谢所有正在阅读这篇论文的读者, 你们给了我继续完善的动力. 本文的源码和编译的 PDF 文件将托管在 \url{https://github.com/JLHwung/MSThesis-PKU}, 欢迎提出修改意见.