\specialchap{引言}
我们以一个虚拟的情景来开篇. 假设一家大型医院收集了许多患者的病历, 院方希望与医学专家、统计专家合作来对病历库的信息进行医学研究. 专家需要知道病历库信息的一些统计特征, 但医院又需要保证每一个患者的隐私不被泄露. 因此, 医院需要一个特殊的数据接入机制来保证专家只能获取统计信息, 而不能获取每一个患者的个体信息. 

在早期, 一种简单的想法是将病历库中患者的身份信息删除, 如姓名、单位等等, 随后由于计算机科学的兴起, 人们开始采取哈希等匿名化方法. 但是, 无论是删除还是匿名化, 患者的隐私仍然可能会泄露, 因为数据库的攻击者完全可以通过其他信息渠道, 尤其是近年兴起的社交网络, 来识别其中某一条具体的病历记录. 比如攻击者可以通过搜索获知某个人的年龄、性别、民族与婚姻状况, 联合这些信息就有可能从大量的病历中直接筛选出这个人的病历, 这样虽然病历记录中没有提供患者的姓名, 但患者的信息实际上已经泄露了. 更有甚者, 如果攻击者获知某个人患有某种罕见病, 例如苯丙酮尿症(Phenylketonuria, 简称 PKU)在中国的发病率为$1/18000$\parencite{shen1986newborn}, 那么通过这样一个信息就可以更容易地从大规模病历中锁定患者个人, 这对患者的隐私权无疑是极大的侵犯.

而事实上, 这样的事情在现实世界中确有发生. 著名的视频分享网站Netflix在2007年公开了部分用户对电影的评级信息, 并举行了预测用户评级的机器学习竞赛. Netflix 对公开数据库中的用户ID等敏感信息都进行了匿名化处理, 但研究人员利用抓取的 IMDb 的用户评分以及 ID, 成功地还原出了该数据库中的部分敏感信息
\parencite{narayanan2008robust}. 两年后四名Netflix用户控告Netflix公司, 其中一项便是公开数据库触犯了视频隐私保护法案(Video Privacy Protection Act), 尽管几个月后Netflix与这四名用户达成了庭外和解, 这依旧在社会引发了关于研究与隐私边界的广泛讨论.

从密码学的观点来看, 匿名化实际上是一种保证数据库语义安全(semantic security, \parencite{goldwasser1984probabilistic})的一种手段. 语义安全早在1984年就已经提出, 语义安全最早基于加密系统, 它指的是对于监听者而言, 获取密文与否都无法增减对明文了解的信息. 对于数据库而言,  它指的是接入统计数据库不能学习到任何关于某个个体的额外信息, 这里的额外信息特指未接入该数据库所不能学习到的信息. 换言之, 如果病历库中有患者张三, 为男性, 那么匿名化保证了接入数据库与未接入数据库都无法获知张三为男性这一信息. 但语义安全最大的问题是无法阻止辅助信息带来的攻击: 我们无法从数据库中获知张三是男性, 但我们可以从其他渠道获知张三的一些信息, 比如说张三是该医院年龄最大的病人, 那么从病历库中按年龄排序就能知道张三是男性这一信息. Netflix 事件同样因为语义安全无法保证数据库个体隐私.

因此, 为了避免这种明显的对隐私的侵犯, 有必要重新定义一种隐私目标: 考虑数据库中的每一个个体, 我们希望无论该个体是否在这个数据库中, 该个体的隐私泄露风险几乎是一样的. 例如, 张三身高1.65米这一信息泄露的概率是10\%, 换言之攻击者通过各种渠道只能以10\%的概率成功断定张三的身高是1.65米, 那么, 我们希望即便张三的病历出现在这个病历库中, 攻击者也还是只能以接近10\%的概率去确定张三的真实身高. 注意到在该隐私目标中, 辅助信息完全不能帮助攻击者去提高自己窃取隐私的成功率, 因此该隐私目标可以防范辅助信息的攻击.

我们将在第\ref{cha:背景}章用差分隐私对上述隐私目标进行严格刻画(\ref{sec:差分隐私定义}小节), 随后讨论组合定理(\ref{sec:组合定理}小节)、精确性与能回答指数多个查询的指数机制(\ref{sec:精确性与查询规模}小节)以及输出合成数据库的网络机制(\ref{sec:合成数据库}小节).  我们在第\ref{cha:理论}章对一大类光滑查询提出了三角机制, 证明了其同时保证隐私性、精确性(\ref{sec:_epsilon_差分隐私机制}小节与\ref{sec:_epsilon_delta_差分隐私机制}小节). 与现有的指数时间代价的机制相比, 三角机制时间代价为多项式, 并对高度光滑的函数拥有比现有机制更好的精确性保证(\ref{sec:与网络机制的性能对比}小节). 随后我们进一步改进三角机制的运行时间, 并利用隐私主成分分析来提高机制的精度(\ref{sec:使用隐私主成分分析改进运行时间}小节). 在第\ref{cha:实验}章, 我们在包含敏感信息的数据库上使用三角机制, 并对随机生成的高斯线性组合查询计算误差(\ref{sec:最坏情形误差}小节), 我们还探讨了误差与一些外生参数的关系(\ref{sec:误差与部分参数关系}小节).
